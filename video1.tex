\documentclass{article}
\usepackage{amsmath,amssymb,amsthm,fullpage,enumerate,bm,hyperref}
\theoremstyle{definition}
\newtheorem{defi}{Definition}
\newtheorem{theo}{Theorem}
\newtheorem{lem}{Lemma}
\newtheorem{prop}{Proposition}
\title{{\bf MODULAR BASICS\\VIDEO 1}}
\begin{document}
\maketitle
In this video series I'll teach you, and myself, the basics of modular forms. Especially the geometric perspective. I do research in modular forms, but often find that I lack sufficient knowledge of the basics.

In this first video, I want to try and understand the structure of the quotient spaces $\Gamma\backslash\mathbb{H}^\ast$ for $\Gamma$ a Fuchsian group of the first kind, as Riemann surfaces. This is an important ingredient in what will begin doing in the next video--namely using the Riemann-Roch theorem to compute the dimension of various spaces of modular forms, and especially spaces of vector-valued modular forms.

The whole series will mainly be based on Miyake for the elliptic modular forms stuff. When it comes to vector-valued modular forms, it will be based on my own notes, a paper by Borcherds, and every paper by Raum.
\section{The Möbius action and the co-cycle $j$}
Recall that we have the Möbius action of $\mathrm{GL}_2(\mathbb{C})$ on the Riemann sphere $\mathbb{P}=\mathbb{C}\cup\{\infty\}$ given by
\[\begin{pmatrix}a&b\\c&d\end{pmatrix}\! z=\frac{az+b}{cz+d},\]
for $z\neq\infty$. Since
\[\lim_{t\to\infty}\frac{ait+b}{cit+d}=\lim_{t\to\infty}\frac{ai+b/t}{ci+d/t}=a/c,\]
we define
\[\begin{pmatrix}a&b\\c&d\end{pmatrix}\!\infty=\frac{a}{c}.\]
Note that we use the convention $a/0=\infty$.

We write for $\alpha=(a,b;c,d)\in\mathrm{GL}_2(\mathbb{C})$
\[a(\alpha)=a,b(\alpha)=b,c(\alpha)=c,\text{ and }d(\alpha)=d.\]
For $\alpha\in\mathrm{GL}_2(\mathbb{C})$ and $z\in\mathbb{P}$ we define
\[j(\alpha,z)=c(\alpha)z+d(\alpha).\]
It is a fact that $j$ is a co-cycle, by which I mean that it satisfies
\[j(\alpha\beta,z)=j(\alpha,\beta z)j(\beta,z),\]
for ``appropriate'' $\alpha$, $\beta$ and $z$. What I mean by appropriate in this case is that
\[\alpha,\beta\in\mathrm{GL}_2(\mathbb{R})\text{ and }z\in\mathbb{C}\setminus\mathbb{R}.\]
It probably works even for $\mathrm{GL}_2(\mathbb{C})$ and $z\in\mathbb{P}$, but I want to avoid gnarly infinities and whatnot.\footnote{We want to let $0\cdot\infty$ be undefined on the Riemann sphere, so $j(\alpha,\beta z)j(\beta,z)$ becomes problematic.}

Anyway, the way to see this, and to simultaneously show that the Möbius action is indeed an action is that
\[\alpha\begin{pmatrix}z\\1\end{pmatrix}=\begin{pmatrix}az+b\\cz+d\end{pmatrix}=j(\alpha,z)\begin{pmatrix}\alpha z\\1\end{pmatrix},\]
which yields that
\[\alpha\beta\begin{pmatrix}z\\1\end{pmatrix}=j(\alpha\beta,z)\begin{pmatrix}(\alpha\beta)z\\1\end{pmatrix},\]
but also
\[\alpha\beta\begin{pmatrix}z\\1\end{pmatrix}=j(\alpha,\beta z)j(\beta, z)\begin{pmatrix}\alpha(\beta z)\\1\end{pmatrix}.\]
The aforementioned conditions on $\alpha$ and $\beta$ and $z$, ensure that
\[j(\alpha\beta,z),j(\alpha,\beta z),j(\beta,z)\neq 0,\]
so that we can conclude that $j$ is a co-cycle, and that the Möbius action is an action.

Before continuing with more interesting things, we move to the more familiar setting of the upper half-plane. Since for $\alpha=(a,b;c,d)\in\mathrm{GL}_2(\mathbb{R})$ and $z\in\mathbb{P}$ we have that
\[\frac{az+b}{cz+d}=\frac{(az+b)(c\overline{z}+d)}{|cz+d|^2}=\frac{ac|z|^2+bd+x(ad+bc)+iy(ad-bc)}{|cz+d|^2},\]
where $z=x+iy$, it holds that
\[\mathrm{Im}(\alpha z)=\frac{\det(\alpha)\mathrm{Im}(z)}{|cz+d|^2}.\]
Hence, for $\alpha\in\mathrm{GL}_2(\mathbb{R})$ with $\det(\alpha)>0$ and $z\in\mathbb{H}$, we have that $\alpha z\in\mathbb{H}$.

We therefore define
\[\mathrm{GL}_2^+(\mathbb{R})=\{\alpha\in\mathrm{GL}_2(\mathbb{R}):\det(\alpha)>0\},\]
and conclude that for $\alpha\in\mathrm{GL}_2^+(\mathbb{R})$ the map
\[\mathbb{H}\ni z\mapsto\alpha z\in\mathbb{H},\]
is a complex analytic automorphism, or in other words, a biholomorphic map (if you prefer the categorical point of view).
\section{Elliptic, parabolic, and hyperbolic elements}
In order to give a geometric definition of a ``cusp'', we first classify the elements in $\mathrm{GL}_2^+(\mathbb{R})$ depending on how the square of the trace relates to the determinant.
\begin{defi}
Let $\alpha\in\mathrm{GL}_2^+(\mathbb{R})$ be non-scalar. If
\[\mathrm{Tr}(\alpha)^2<4\det(\alpha),\]
then $\alpha$ is called {\bf elliptic}, and if
\[\mathrm{Tr}(\alpha)^2=4\det(\alpha),\]
then $\alpha$ is called {\bf parabolic}, and if
\[\mathrm{Tr}(\alpha)^2>4\det(\alpha),\]
then $\alpha$ is called {\bf hyperbolic}.
\end{defi}
I frankly don't know where the terminology comes from, but I guess it will become clear as I mature.

If $\alpha\in\mathrm{GL}_2^+(\mathbb{R})$ is scalar, then $\alpha=aI$ and
\[\mathrm{Tr}(\alpha)^2=(2a)^2=4a^2=4\det(\alpha),\]
so we could argue that scalar elements should be called parabolic. But then the terminology clashes with the following theorem, I think.\footnote{I will be honest with my own knowledge gaps in this series. It will be important to be able to refer to misunderstandings further on.}
\begin{theo}
	Let $\alpha\in\mathrm{GL}_2^+(\mathbb{R})$ be non-scalar. Then
	\begin{itemize}
		\item $\alpha$ is elliptic iff $\alpha$ has the fixed points $z_0$ and $\overline{z_0}$ for some $z_0\in\mathbb{H}$
		\item $\alpha$ is parabolic iff $\alpha$ has a unique fixed point on $\mathbb{R}\cup\{\infty\}$
		\item $\alpha$ is hyperbolic iff $\alpha$ has two distinct fixed points on $\mathbb{R}\cup\{\infty\}$.
	\end{itemize}
\end{theo}
\begin{proof}
	Miyake uses an in-line proof, but I don't like that, so here goes my reformulation.

	Say that $\alpha=(a,b;c,d)$. If $c=0$, then $\mathrm{Tr}(\alpha)=a+d$ and $\det(\alpha)=ad$, so
	\[\mathrm{Tr}(\alpha)^2-4\det(\alpha)=(a+d)^2-4ad=(a-d)^2.\]
	Hence $\alpha$ is parabolic iff $a=d$, and hyperbolic iff $a\neq d$.

	If $\alpha$ is parabolic, we have that $\alpha z=z$ iff
	\[z+\frac{b}{a}=z.\]
	This is evidently only the case for $z=\infty$.

	If $\alpha$ is hyperbolic, we have that $\alpha z=z$ iff
	\[\frac{a}{d}z+\frac{b}{d}=z.\]
	This equation has both finite and non-finite solutions. Clearly $z=\infty$ is a solution. The finite solution is
	\[z=\frac{b}{d}\frac{d}{d-a}=\frac{b}{d-a}.\]

	If $c\neq 0$, then $\alpha\infty=a/c\neq\infty$, so $\infty$ cannot be a fixed point. We have that $\alpha z=z$ iff
	\begin{align}
		\frac{az+b}{cz+d}=z,
		\label{eq:fixedpt1}
	\end{align}
	If $z=-d/c$, then the left-hand side is equal to $\infty$, so this cannot be a solution. Hence $cz+d\neq 0$ and we obtain that~\eqref{eq:fixedpt1} is equivalent to (recall that $c\neq 0$)
	\[z^2+\frac{d-a}{c}z-\frac{b}{c}=0.\]
	We solve this equation
	\[z^2+\frac{d-a}{c}z-\frac{b}{c}=(z+\frac{d-a}{2c})^2-\frac{1}{4c^2}((d-a)^2+4bc).\]
	We notice that
	\[\mathrm{Tr}(\alpha)^2-4\det(\alpha)=(a+d)^2-4(ad-bc)=(a-d)^2+4bc,\]
	so that if $\mathrm{Tr}(\alpha)^2-4\det(\alpha)=0$ there is the unique fixed point $(a-d)/2c$, and if $\mathrm{Tr}(\alpha)^2-4\det(\alpha)>0$ there are two real conjugate fixed points
	\[z_{1,2}=\frac{1}{2c}(a-d\pm\sqrt{\mathrm{Tr}(\alpha)^2-4\det(\alpha)}),\]
	and if $\mathrm{Tr}(\alpha)^2-4\det(\alpha)<0$ there are two complex conjugate fixed points
	\[z_0,\overline{z_0}=\frac{1}{2c}(a-d\pm i\sqrt{4\det(\alpha)-\mathrm{Tr}(\alpha)^2}).\]

	So, if $\alpha$ is elliptic, then it has to be the case that $c\neq 0$, and then we have two complex conjugate solutions. If there are two complex conjugate fixed points, we see from the above that the only case in which this occurs is when $\alpha$ is elliptic.

	If $\alpha$ is parabolic it can be the case that $c=0$ or $c\neq 0$. If $c=0$, then we have the fixed point $\infty$. If $c\neq 0$, then there is the unique fixed point $(a-d)/2c$. Conversely, suppose there is a unique fixed point. Then we see from the above that this can only occur when $\alpha$ is parabolic.

	If $\alpha$ is hyperbolic, it can be the case that $c=0$ or $c\neq 0$. If $c=0$, then we have the distinct fixed points $\infty$ and $b/(d-a)$. If $c\neq 0$, then we have two (distinct) real conjugate fixed points. Conversely, if there are two distinct (extended) real fixed points, then we see from the above that this only occurs when $\alpha$ is hyperbolic.
\end{proof}
\section{Stabilizers}
We are now going to spend some time on understanding stabilizer subgroups of $\mathrm{GL}_2^+(\mathbb{R})$.

Recall that for a group $G$ acting on a set $X$, we use the notation
\[G_x=\{g\in G:gx=x\},\]
for the stabilizer group of $x\in X$, and the notation
\[Gx=\{gx:g\in G\},\]
for the orbit of $x\in X$.

The stabilizer subgroups of $\mathrm{GL}_2^+(\mathbb{R})_x$ for $x\in\mathbb{R}\cup\{\infty\}$ contain scalar, parabolic, and hyperbolic elements. To distinguish between these, we introduce the following notation (here $x\neq x'\in\mathbb{R}\cup\{\infty\}$)
\begin{align*}
	\mathrm{GL}_2^+(\mathbb{R})_x^{\mathrm{p}}&=\{\alpha\in\mathrm{GL}_2^+(\mathbb{R})_x:\alpha\text{ is parabolic or scalar}\}\\
	\mathrm{SL}_2⁺(\mathbb{R})_x^\mathrm{p}&=\mathrm{SL}_2(\mathbb{R})\cap\mathrm{GL}_2^+(\mathbb{R})_x^\mathrm{p}\\
	\mathrm{GL}_2^+(\mathbb{R})_{x,x'}&=\mathrm{GL}_2^+(\mathbb{R})_x\cap\mathrm{GL}_2^+(\mathbb{R})_{x'}\\
	\mathrm{SL}_2^+(\mathbb{R})_{x,x'}&=\mathrm{SL}_2(\mathbb{R})\cap\mathrm{GL}_2^+(\mathbb{R})_{x,x'}.
\end{align*}
We recall that the special orthogonal group of degree $2$ is defined by
\[\mathrm{SO}_2(\mathbb{R})=\Big\{\!\begin{pmatrix}\cos\theta&\sin\theta\\-\sin\theta&\cos\theta\end{pmatrix}\!:0\leq\theta<2\pi\Big\}.\]

When studying $\Gamma\backslash\mathbb{H}^\ast$ we shall make use of the following three lemmata.
\begin{lem}
	It holds that
	\begin{enumerate}
		\item $\mathrm{GL}_2^+(\mathbb{R})_i=\mathbb{R}^\times\cdot\mathrm{SO}_2(\mathbb{R})$,
		\item $\mathrm{GL}_2^+(\mathbb{R})_\infty=\Big\{\begin{pmatrix}a&b\\0&d\end{pmatrix}:a,b\in\mathbb{R}^\times,b\in\mathbb{R},ad>0\Big\}$,
		\item $\mathrm{GL}_2^+(\mathbb{R})_\infty^\mathrm{p}=\Big\{\begin{pmatrix}a&b\\0&a\end{pmatrix}:a\in\mathbb{R}^\times,b\in\mathbb{R}\Big\}$,
		\item $\mathrm{GL}_2^+(\mathbb{R})_{0,\infty}=\Big\{\begin{pmatrix}a&0\\0&d\end{pmatrix}:a,d\in\mathbb{R}^\times,ad>0\Big\}$.
	\end{enumerate}
	Furthermore,
	\begin{enumerate}
		\item the group $\mathrm{GL}_2^+(\mathbb{R})_z$ with $z\in\mathbb{H}$ is $\mathrm{SL}_2(\mathbb{R})$ conjugate to $\mathrm{GL}_2^+(\mathbb{R})_i$;
		\item the group $\mathrm{GL}_2^+(\mathbb{R})_x^\mathrm{p}$ with $x\in\mathbb{R}\cup\{\infty\}$ is $\mathrm{SL}_2(\mathbb{R})$ conjugate to $\mathrm{GL}_2^+(\mathbb{R})_\infty^\mathrm{p}$;
		\item and the group $\mathrm{GL}_2^+(\mathbb{R})_{x,x'}$ with $x\neq x'\in\mathbb{R}\cup\{\infty\}$ is $\mathrm{SL}_2(\mathbb{R})$ conjugate to $\mathrm{GL}_2^+(\mathbb{R})_{\infty,0}$.
	\end{enumerate}
\end{lem}
\begin{proof}
	For 1., notice that if $\alpha\in\mathrm{GL}_2^+(\mathbb{R})$ then $\det(\alpha)^{-1/2}I\alpha\in\mathrm{SL}_2(\mathbb{R})$. Say $(a,b;c,d)=\alpha\in\mathrm{SL}_2(\mathbb{R})$ and $\alpha i=i$. Then
	\[ai+b=-c+di,\]
	and $ad-bc=1$. So since all entries are real, we find
	\[a=d\text{ and }b=-c,\]
	and $a^2+b^2=1$. Therefore $a=\cos(\theta)$ and $b=\sin(\theta)$ for some $0\leq\theta<2\pi$. Thus $\alpha\in\mathrm{SO}_2(\mathbb{R})$. The reverse direction is obvious.

	Hence, we find that if $\alpha\in\mathrm{GL}_2^+(\mathbb{R})_i$, then $\alpha=\det(\alpha)^{-1/2}\beta$ where $\beta\in\mathrm{SO}_2(\mathbb{R})$. Since $kIi=ki/k$, scalar matrices fix $i$, and since matrices in $\mathrm{SO}_2(\mathbb{R})$ also fix $i$, the reverse direction is clear.
\end{proof}
Recall that the zentrum of an element $\alpha\in G$ in a group $G$ is defined to be
\[\mathrm{Z}(\alpha)=\{\beta\in G:\alpha\beta=\beta\alpha\}.\]
For elements in $\mathrm{GL}_2^+(\mathbb{R})$, we view them as elements in the supergroup $\mathrm{GL}_2^(\mathbb{R})$.

The following lemma will be left unproven, unless it turns out we need it later.
\begin{lem}
	Let $z\in\mathbb{H}$ and $x\neq x'\in\mathbb{R}\cup\{\infty\}$. For $\alpha\in\mathrm{GL}_2^+(\mathbb{R})$ non-scalar, we have
	\begin{enumerate}
		\item If $\alpha\in\mathrm{GL}_2^+(\mathbb{R})_z$ then $\mathrm{Z}(\alpha)=\mathrm{GL}_2^+(\mathbb{R})_z$.
		\item If $\alpha\in\mathrm{GL}_2^+(\mathbb{R})_x^\mathrm{p}$ then $\mathrm{Z}(\alpha)=\mathrm{GL}_2^+(\mathbb{R})_x^\mathrm{p}$.
		\item If $\alpha\in\mathrm{GL}_2^+(\mathbb{R})_{x,x'}$ then $[\mathrm{Z}(\alpha):\mathrm{Z}(\alpha)\cap\mathrm{GL}_2^+(\mathbb{R})]=2$ and $\mathrm{Z}(\alpha)\cap\mathrm{GL}_2^+(\mathbb{R})=\mathrm{GL}_2^+(\mathbb{R})_{x,x'}$.
	\end{enumerate}
\end{lem}
No proof also goes for the following lemma. We recall that for $G$ a subgroup of $\mathrm{GL}_2^+(\mathbb{R})$, we define the normalizer to be
\[\mathrm{N}(G)=\{\alpha\in\mathrm{GL}_2(\mathbb{R}):\alpha G\alpha^{-1}=G\}.\]
\begin{lem}
	Let $z\in\mathbb{H}$ and $x\neq x'\in\mathbb{R}\cup\{\infty\}$. Let $G$ be equal to $\mathrm{GL}_2^+(\mathbb{R})_z$, $\mathrm{GL}_2^+(\mathbb{R})_x^\mathrm{p}$, or $\mathrm{GL}_2^+(\mathbb{R})_{x,x'}$. Then $[\mathrm{N}(G):\mathrm{N}(G)\cap\mathrm{GL}_2^+(\mathbb{R})]=2$ and
	\begin{enumerate}
		\item $\mathrm{N}(G)\cap\mathrm{GL}_2^+(\mathbb{R})=G$ if $G=\mathrm{GL}_2^+(\mathbb{R})_z$,
		\item $\mathrm{N}(G)\cap\mathrm{GL}_2^+(\mathbb{R})=\mathrm{GL}_2^+(\mathbb{R})_x$ if $G=\mathrm{GL}_2^+(\mathbb{R})_x^\mathrm{p}$,
		\item $[\mathrm{N}(G)\cap\mathrm{GL}_2^+(\mathbb{R}):G]=2$ if $G=\mathrm{GL}_2^+(\mathbb{R})_{x,x'}$.
	\end{enumerate}
\end{lem}
Finally we have the following lemma. We will prove this though.
\begin{lem}
	Any two distinct elliptic elements in $\mathrm{GL}_2^+(\mathbb{R})_z$ with $z\in\mathbb{H}$ are non-conjugate in $\mathrm{GL}_2^+(\mathbb{R})$.

	If two distinct parabolic elements in $\mathrm{GL}_2^+(\mathbb{R})$ are conjugate by a matrix of negative detemrinant, then they are not conjugate in $\mathrm{GL}_2^+(\mathbb{R})$.
\end{lem}
\begin{proof}
\end{proof}
\section{Something very short on geodesics (hurhur)}
{\it This part can be visualized very nicely. Worth at least doing some basic pictures of geodesics on the unit disk and then mapping them via $\rho$ back to $\mathbb{H}$.}

This section will be very cursory, but at the very least we need to introduce some vocabulary concerning geodesics on $\mathbb{H}$.

On $\mathbb{H}$ the line element $\mathrm{d}s$ is defined by
\[\mathrm{d}s^2=\frac{\mathrm{d}x^2+\mathrm{d}y^2}{y^2},\]
and the volume element $\mathrm{d}v$ is defined by
\[\mathrm{d}v=\frac{\mathrm{d}x\mathrm{d}y}{y^2}.\]
I'm not good with differential geometry, so I take this as definitions for how to calculate arc-length and volume on $\mathbb{H}$. (Note that the line element and volume element are $\mathrm{GL}_2^+(\mathbb{R})$ invariant, and so they give arc length and volume also on quotient spaces.)

If $A$ is some region in $\mathbb{H}$ we therefore define its volume to be
\[v(A)=\int_{A}\mathrm{d}v,\]
and if $\phi:[0,1]\to\mathbb{H}$ is injective continuous and smooth almost everywhere, then we define the length of $\mathrm{im}(\phi)$ to be
\[l(\mathrm{im}(\phi))=\int_0^1\frac{\sqrt{x'(t)^2+y'(t)^2}}{y(t)}\mathrm{d}t.\]
What is cool is:
\begin{prop}
	Given $z_1,z_2\in\mathbb{H}$, there is a shortest curve connecting $z_1$ and $z_2$.
\end{prop}
The shortest curve is called the geodesic, and we define the distance $d(z_1,z_2)$ between $z_1$ and $z_2$ to be the length of the geodesic.
\section{Fuchsian groups}
Before talking about Fuchsian group in full generality. Let us discuss $\mathrm{SL}_2(\mathbb{R})$ as a topological space.

For any $z=x+iy\in\mathbb{H}$ put
\[h_z=\frac{1}{\sqrt{y}}\begin{pmatrix}y&x\\0&1\end{pmatrix}\in\mathrm{SL}_2(\mathbb{R}).\]
Let $\alpha\in\mathrm{SL}_2(\mathbb{R})$ and $z=\alpha i$. Then
\[h_z^{-1}\alpha i=h_z^{-1}z=\sqrt{y}\begin{pmatrix}1&-x\\0&y\end{pmatrix}z=\frac{z-x}{y}=i.\]
Hence, since $h_z^{-1}\alpha\in\mathrm{SL}_2(\mathbb{R})$ we have that $h_z^{-1}\alpha\in\mathrm{SO}_2(\mathbb{R})$. Hence for some $0\leq\theta<2\pi$ it holds that
\[h_z^{-1}\alpha=k_\theta\overset{\Delta}{=}\begin{pmatrix}\cos\theta&\sin\theta\\-\sin\theta&\cos\theta\end{pmatrix}.\]
So $\alpha=h_zk_\theta$. Now suppose that
\[h_{z'}k_{\theta'}=h_zk_\theta.\]
Then $z'=h_{z'}k_{\theta'}i=h_zk_\theta i=z$ and similarly $\theta'=\theta$. This shows that there is a bijection
\[f:\mathrm{SL}_2(\mathbb{R})\ni (h_zk_\theta)\mapsto(z,\theta)\mathbb{H}\times\mathbb{C}^1.\]
It's easy to verify that the inverse to this bijection is entry-wise continuous, and this allows to define a topology for $\mathrm{SL}_2(\mathbb{R})$ by taking the open sets to be $f^{-1}(U)$ for $U$ open in $\mathbb{H}\times\mathbb{C}^1$. (Maybe do some more on this when prepping the "slides".)

Suppose now that we have a topological group $\Gamma$ acting on a locally compact space $X$. Then we say that it acts properly discontinuously if for any compact subsets $A,B\subseteq X$ it holds that
\[|\{\gamma\in\Gamma:\gamma A\cap B\neq\emptyset\}|<\infty.\]
[ILLUSTRATE!]
Clearly, this must be some kind of condition relating to making $\Gamma\backslash X$ Hausdorff. Indeed it is, and since Riemann surfaces are manifolds, and therefore second countable Hausdorff; this will be crucial in what follows.

Let $G$ be a second countable locally compact topological group acting transitively on locally compact Hausdorff space $X$. Let also $\Gamma$ be a subgroup of $G$. Then if all stabilizers of elements of $G$ are compact, the 
\section{Fundamental domains}
{\it They exist.}
\section{Quotient groups}
\section{$\Gamma\backslash\mathbb{H}^\ast$ as a Riemann surface}
\section{Fuchsian groups of the first kind}
\end{document}
